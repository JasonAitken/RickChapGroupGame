\documentclass[10pt,a4paper,titlepage]{article}
\usepackage[utf8]{inputenc}
\usepackage{amsmath}
\usepackage{amsfonts}
\usepackage{amssymb}
\usepackage{lmodern}
%Sets the margin to 1 inch
\usepackage[margin=1in]{geometry}

%used for imagery and captions
\usepackage{subcaption}
\usepackage{graphicx}
\usepackage[export]{adjustbox}

%'codify' text for snippets
\usepackage{xcolor}
\definecolor{codegray}{gray}{0.9}
\newcommand{\code}[1]{\colorbox{codegray}{\texttt{#1}}}

\author{Michael Ransby\\300447412}
\title{Software Engineering 225\\\textit{"Levels as Plugins"}}

\begin{document}
\maketitle
\section{Implementation}
\subsection{Design Patterns}
\subsection{Frameworks}

\section{Usage}
\subsection{Installation}
Installing plug-ins (known as "levels" from now) is as easy as placing the \code{*.zip} files in the \code{assets/zips/} directory. The zip file should contain only the \code{*.json} file inside.\\
You may also place \code{*.json} files in this directory but it is not recommended.

\subsection{Utilisation}


\end{document}